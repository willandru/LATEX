
\documentclass[journal,transmag]{IEEEtran}
\hyphenation{op-tical net-works semi-conduc-tor}




% *** GRAPHICS RELATED PACKAGES ***
%
\ifCLASSINFOpdf
   \usepackage[pdftex]{graphicx}
  % declare the path(s) where your graphic files are
  % \graphicspath{{../pdf/}{../jpeg/}}
  % and their extensions so you won't have to specify these with
  % every instance of \includegraphics
  % \DeclareGraphicsExtensions{.pdf,.jpeg,.png}
\else
  % or other class option (dvipsone, dvipdf, if not using dvips). graphicx
  % will default to the driver specified in the system graphics.cfg if no
  % driver is specified.
  % \usepackage[dvips]{graphicx}
  % declare the path(s) where your graphic files are
  % \graphicspath{{../eps/}}
  % and their extensions so you won't have to specify these with
  % every instance of \includegraphics
  % \DeclareGraphicsExtensions{.eps}
\fi
% graphicx was written by David Carlisle and Sebastian Rahtz. It is
% required if you want graphics, photos, etc. graphicx.sty is already
% installed on most LaTeX systems. The latest version and documentation
% can be obtained at: 
% http://www.ctan.org/pkg/graphicx
% Another good source of documentation is "Using Imported Graphics in
% LaTeX2e" by Keith Reckdahl which can be found at:
% http://www.ctan.org/pkg/epslatex
%
% latex, and pdflatex in dvi mode, support graphics in encapsulated
% postscript (.eps) format. pdflatex in pdf mode supports graphics
% in .pdf, .jpeg, .png and .mps (metapost) formats. Users should ensure
% that all non-photo figures use a vector format (.eps, .pdf, .mps) and
% not a bitmapped formats (.jpeg, .png). The IEEE frowns on bitmapped formats
% which can result in "jaggedy"/blurry rendering of lines and letters as
% well as large increases in file sizes.
%
% You can find documentation about the pdfTeX application at:
% http://www.tug.org/applications/pdftex








\begin{document}

\title{\textsc{ELASTICIDAD DE UN RESORTE}}

\author{
\IEEEauthorblockN{William A. Gómez,Carlos J. Tabares,Juan M. Leon,
Julian D. Colmenares, andLuis A. Cañon,}
\IEEEauthorblockA{Pontificia Universidad Javeriana, Bogotá, Colombia}
\IEEEauthorblockA{Laboratorio de Fluidos y Termodinámica}
\IEEEauthorblockA{Grupo IV}
\thanks{PUJ}
}
% The paper headers
\markboth{Elasticidad. Agosto 4~2021}%
{Shell \MakeLowercase{\textit{et al.}}: Bare Demo of IEEEtran.cls for IEEE Transactions on Magnetics Journals}
\IEEEtitleabstractindextext{%

	\begin{abstract}
	Con la intension de estudiar y analizar el comportamiento de los diferentes materiales de la naturaleza, hemos realizado unos experimetos de esfuerzo-deformación sobre un caucho o banda elastica comun, con el fin de contrastar el comportamiento de los materiales rigidos y solidos con los materiales elasticos por medio de la ecuacion de la Ley de Hook y los diagramas de esfuerzo-deformacion. En este laboratorio se tomaron varias muestras iguales de bandas elasticas con dimensiones 14mm x 2mm x 2mm y se sometieron a a los mismos experimentos, esto con el fin de determinar el modulo de Young, el límite de elasticidad, así como los respectivos calculos de la desviación estandar, incertidumbre experimental y error relativo de los datos medidos experimentalmente.
	\end{abstract}
	
	\begin{IEEEkeywords}
	Elasticidad, Esfuerzo, Deformación, Banda elástica, Caucho, Ley de Hook, Módulo de Young, Límite elasticidad.
	\end{IEEEkeywords}}

\maketitle
\IEEEdisplaynontitleabstractindextext
\IEEEpeerreviewmaketitle


\section{Introduction}
	\IEEEPARstart{T}{his} demo file is intended to serve as a 		``starter file''
	for IEEE \textsc{Transactions on Magnetics} journal papers 	produced under \LaTeX\ using
	IEEEtran.cls version 1.8b and later.

	I wish you the best of success.




%%%%%%%%%%%%%%%%%%%%%%%%%%METODOLOGIA
\section{Metodología}

\subsection{Límite de elasticidad}
Para la primera parte del laboratorio queríamos estimar el límite de deformación de un caucho, es decir queríamos averigar el esfuerzo máximo al cual pueda estirarse sin deformarse.
 Para esto , utilizamos un caucho o banda elastica común, como el de la Figura \ref{fig1}, y con la ayuda de un papel milimetrado cortamos 2 cauchos de 14 milimetros de longitud y aproximadamente 1.8 milimetros de sección transversal. 

\begin{figure}[!h]
\center
\includegraphics[width=3cm]{caucho.jpeg}
\includegraphics[width=3cm]{medida.jpeg}
\caption{Banda elástica común de 1.5mm de groso aproximadamente.}
\label{fig1}
\end{figure}

Hallamos el límite de elasticidad para estos 2 cauchos agregando grupos de monedas a vasos plasticos  de montajes como el de la Figura \ref{fig2} y midiendo su deformación con la ayuda de un papel milimétrico. El montaje se realizo atando una cuerda a un vaso de plastico y amarrando la cuerda al extremo del caucho con la ayuda de un hilo de tension. El montaje se colgo del mango de una puerta y se acomodo por detras del caucho una regla de papel milimetrado para que el extremo inicial del caucho se encontrara al nivel 0 del papel milimetrado (0 mm) y se pudieran hacer las distintas pruebas y mediciones varias veces.
\begin{figure}[!h]
\center
\includegraphics[width=5cm]{montaje.jpeg}
\caption{Límite del comportamiento elástico. Se graficó el cambio de la longitud del caucho en función de la masa soportada (distinta cantidad de monedas).}
\label{fig2}
\end{figure}

Para la prueba se utilizaron monedas de 1 "penique" Británico, que tienen una masa de exacta de 3.465 gramos, como se observa en la Figura \ref{fig3}.

\begin{figure}[!h]
\center
\includegraphics[width=3cm]{penny.png}
\caption{Moneda de "1 penique" del Reino Unido. Pesa exactamente 3.564 ± 0.000005 gramos.}
\label{fig3}
\end{figure}

Para calcular el límite de elasticidad, se agregaron monedas de a grupos de 4, para que se pudiera evidenciar el cambio en la longitud del caucho (delta l) y se registro esta deformación hasta que fue posible llevar a cabo el experimento.


\subsection{Esfuerzó-Deformación}
Se dividió el caucho de 14mm en 3 secciones iguales y se hizo seguimiento a la deformación ocurrida en cada segmento, al aplicarle una deformación de tension en el eje de la fuerza de la gravedad (debido al peso de las monedas). Se tomaron 2 cauchos iguales y se dividieron en 3 secciones equidistantes, se agregaron grupos de 4 monedas hasta completar un total de 20 monedas en el vaso y se registro en cada paso el desplazamiento de las marcas con la ayuda del papel milimetrado. 

Los experimentos se repitieron 3 veces en cada caucho, estos datos se alamcenaron en matrices y se calcularon los correspondientes valores del desplazamiento de las marcas (delta l), calculando el promedio de los tres experimentos con su desviacion estandar, error absoluto y relativo, para cada uno de los 2 cauchos.



\subsection{Análisis de datos}

Con los datos recolectados en las anteriores 2 etapas nos dispusimos a modelar los datos utilizando las ecuaciones de esfuerzo y deformación obtenidas por la Ley de Hook con el fin principal de determinar el módulo de Young del caucho.

Todos los datos se recolectaron durante el experimento utilizando una hoja de cálculo de Excel. Despues se pasaron los datos a Matlab y se organizaron en matrices correspondientes, esto con el fin de realizar los calculos de laboratorio de una manera más controlada, además se utilizó  este software para graficar los resultados.

%%%%%%%%%%%%%%%%%%%%%%%%%%%%%%%%%%%%%%%%%%%%%%%%%%%%
%%%%%%%%%%%%%%%%%%%%%%%%%%RESULTADOS
\section{Resultados}
\subsection{Límite de elasticidad}
\begin{figure}[!h]
\center
\includegraphics[width=10cm]{lomas.jpg}
\caption{Simulation results for the network.}
\label{figr}
\end{figure}



\subsection{Esfuerzó-Deformación}
\begin{figure}[!h]
\center
\includegraphics[width=10cm]{esde1.jpg}
\caption{Simulation results for the network.}
\label{figs}
\end{figure}
\begin{figure}[!h]
\center
\includegraphics[width=10cm]{esde2.jpg}
\caption{Simulation results for the network.}
\label{figs}
\end{figure}
\subsection{Análisis de datos}


%%%%%%%%%%%%%%%%%%%%%%%%%%%%%%%%%%%%%%%%%%%%%%%%%%%%RESULTADOS

\section{Análisis}

\section{Discusión}


\section{Conclusion}
	The conclusion goes here.


\appendices
\section{Proof of the First Zonklar Equation}
	Appendix one text goes here.

\section{}
	Appendix two text goes here.

\ifCLASSOPTIONcaptionsoff
  \newpage
\fi



\begin{thebibliography}{1}

	\bibitem{IEEEhowto:kopka}
	H.~Kopka and P.~W. Daly, \emph{A Guide to \LaTeX}, 3rd~ed.\hskip 1em plus
	  0.5em minus 0.4em\relax Harlow, England: Addison-Wesley, 1999.

\end{thebibliography}


\end{document}


